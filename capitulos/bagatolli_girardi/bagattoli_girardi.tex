\chapterimage{cap_bagattoli}

\chapterauthor{Ricardo Bagatoli e Daniel Girardi}
\chapter{Resistores não ôhmicos: Ensinando eletricidade a partir de uma perspectiva de eletrônica Aplicada}

\section{Apresentação}
Este roteiro faz parte do produto educacional que foi desenvolvido no formato de oficina para ensinar conceitos de eletricidade. É composto por quatro etapas. Cada etapa tem duração média de 4 horas e apresenta conceitos teóricos e atividades práticas sobre eletricidade, com ênfase nos resistores não-ôhmicos. O objetivo desse roteiro é auxiliar o professor na aplicação da oficina, com o intuito de facilitar a compreensão dos conceitos físicos pelos estudantes utilizando métodos não tradicionais de ensino, através da experimentação, coleta e análise de dados utilizando recursos tecnológicos.

No início de cada etapa têm-se uma sugestão de tempo (divididos em teoria e prática), os objetivos, materiais utilizados e as atividades a serem desenvolvidas. Ao decorrer da oficina, espera-se que o aluno faça a apropriação dos conceitos físicos que versam sobre o tema e aplique-os na construção de uma estação meteorológica.


\section{Para o professor}
Para o desenvolvimento da oficina, não é pré-requisito um conhecimento avançado em eletrônica e programação, porém, é de extrema importância que conheça previamente os recursos tecnológicos que serão utilizados. Neste roteiro você terá uma pequena introdução de cada ferramenta tecnológica que será utilizada, com exemplos e atividades práticas. 
	
Cada etapa possui um conjunto de atividades a serem desenvolvidas, sendo:

\begin{itemize}
    \item Primeira etapa: contém informações e atividades sobre a utilização das ferramentas de medição e prototipagem e uma abordagem teórica e prática sobre circuitos simples. 
    \item Segunda etapa: é específica para a configuração e programação do Arduino, que será utilizado como recurso para a captação dos dados obtidos pelos sensores. 
    \item Terceira etapa: é feito o estudo do circuito divisor de tensão, que será utilizado na montagem do circuito com LDR e termistor. Nessa etapa será feita a calibragem dos sensores e o tratamento das informações obtidas por eles. 
    \item Quarta etapa: será construída uma estação meteorológica utilizando as ferramentas estudadas durante a oficina.
\end{itemize}

\section{Materiais utilizados}
Os kits para a aplicação da oficina serão compostos de:
\begin{itemize}
    \item 10 kits de prototipagem, cada kit contendo:
    \begin{itemize}
        \item 1 placa de ensaio (protoboard)
        \item 1 placa de prototipagem (Arduino)
        \item 1 cabo USB
        \item 1 resistor LDR
        \item 1 termistor PTC
        \item 1 termistor NTC
        \item 1 LED
        \item 1 motor elétrico DC – 5V
    \end{itemize}
    \item 10 multímetros
    \item 10 computadores   
\end{itemize}


\section{Estrutura da oficina}
\subsection{Primeira etapa: Introdução à eletricidade e ferramentas de medição}

\subsubsection{Como operar o multímetro}


\textbf{Tempo estimado:} 60minutos divididos entre  20 minutos de teoria e 40 minutos de prática. 


\textbf{Objetivo:} Compreender o funcionamento do multímetro e como utilizá-lo para medir grandezas elétricas.


\textbf{Materiais utilizados:} Bateria, lâmpadas 12 V e resistores.


\textbf{Atividade a ser desenvolvida: }

\Large
\textbf{Teoria: }

\normalsize
\begin{itemize}
    \item Fazer uma breve introdução sobre as 3 partes principais do multímetro (Display, chave seletora e conectores das pontas de prova nos bornes).
    \item Explicar como utilizar as escalas para medir tensão, corrente e resistência, diferenciando tensão com corrente contínua e alternada.
    \item Explicar como fazer a leitura das grandezas físicas utilizando os múltiplos e submúltiplos e interpretando os valores que aparecem no display durante as medições.
    \item Explicar como realizar teste de continuidade utilizando o multímetro.
    \item Explicar como se obter o valor da resistência utilizando código de cores do resistor.
\end{itemize}

\Large
\textbf{Prática:}

\normalsize
\begin{itemize}
    \item Realizar medidas de tensão (ligando o multímetro em paralelo) em corrente contínua, utilizando baterias, selecionando a escala correta e interpretando os valores obtidos.
    
    \item Realizar medidas de corrente elétrica (ligando o multímetro em série com o circuito) utilizando baterias, lâmpadas e resistores.
    \item Realizar medidas de tensão em corrente alternada utilizando as tomadas da sala.
    \item Realizar medidas de resistência elétrica, confrontando os valores obtidos utilizando código de cores com os valores obtidos no multímetro, verificando a faixa de tolerância de cada resistor.
    
\end{itemize}

\Large
\textbf{Atividades práticas:}

\normalsize

\textbf{Medindo tensão corrente contínua:} Distribua as baterias para os alunos e com a chave seletora do multímetro na posição 20V DC (para baterias ou fontes até 20V) coloque a ponta de prova vermelha no polo positivo da bateria e a preta no polo negativo (Figura \ref{fig:bagattoli_1}).

\begin{figure}[h!]
    \centering
    \includegraphics[width=0.65\textwidth]{capitulos/bagatolli_girardi/figuras/1.jpg}
    \caption{Medindo a tensão de uma bateria de 3V em corrente contínua.}
    \label{fig:bagattoli_1}
\end{figure}

\textbf{Medindo tensão corrente alternada:} Com a chave seletora do multímetro na posição 600V AC, coloque a ponta de prova vermelha em um dos orifícios da tomada e a preta no outro (Figura \ref{fig:bagattoli_2}). (Orienta-se que, por questões de segurança, esta parte da atividade seja feita somente pelo professor).

\begin{figure}[h!]
    \centering
    \includegraphics[width=0.65\textwidth]{capitulos/bagatolli_girardi/figuras/2.jpg}
    \caption{Medindo tensão de uma rede 220V em corrente alternada.}
    \label{fig:bagattoli_2}
\end{figure}


\textbf{Medindo corrente contínua:} Utilizando um resistor de $320\Omega$ e uma fonte (utilizado a saída de$5V$ do Arduino) associe o amperímetro em série com o multímetro, ligando no $5V$ e {\bf GND} do Arduino (Figura \ref{fig:bagattoli_3}). Selecione a escala de $20mA$ (se utilizar outro resistor ou a saída de $3,3V$ certifique-se que a corrente que irá passar não supera $20mA$, caso contrário selecione outra escala.

\begin{figure}[h!]
    \centering
    \includegraphics[width=0.65\textwidth]{capitulos/bagatolli_girardi/figuras/3.png}
    \caption{Medindo corrente contínua com o multímetro.}
    \label{fig:bagattoli_3}
\end{figure}


{\bf Medindo resistência elétrica:} Distribua para cada aluno um conjunto de resistores de valores variados e peça para que calculem o valor da resistência utilizando a tabela de resistores (Figura \ref{fig:bagattoli_4}). Considere que a primeira e a segunda faixa são os dígitos significativos, a terceira faixa é o fator de multiplicação e a quarta faixa é a tolerância referente ao valor medido. Por exemplo, um resistor que possui as faixas marrom, vermelho, vermelho e ouro, possui uma resistência de $1200 \Omega$ com uma tolerância de $\pm 60 \Omega$.

\begin{figure}[h!]
    \centering
    \includegraphics[width=0.65\textwidth]{capitulos/bagatolli_girardi/figuras/4.png}
    \caption{Referência de cores para cálculo da resistência de resistores.}
    \label{fig:bagattoli_4}
\end{figure}


Após calcularem os valores teóricos das resistências, selecione no multímetro a função ohmímetro com a chave seletora na escala que contempla o valor teórico. Para o resistor de faixas marrom, vermelho, vermelho e ouro, a escala selecionada deve ser de $2000 \Omega$ (Figura \ref{fig:bagattoli_5}), pois de acordo com as escalas que o multímetro apresenta ($200$, $2K$, $20K$, $200K$ e $2000K$) essa é a que mais se aproxima.

\begin{figure}[h!]
    \centering
    \includegraphics[width=0.65\textwidth]{capitulos/bagatolli_girardi/figuras/5.jpg}
    \caption{Chave seletora em $2000\Omega$.}
    \label{fig:bagattoli_5}
\end{figure}

\subsubsection{
Como utilizar a protoboard
}


{\bf Tempo estimado:} 60 minutos divididos entre 15 minutos de teoria e 45 minutos de prática.



{\bf Objetivo:} Compreender o funcionamento da protoboard utilizando o multímetro para identificar as linhas de alimentação e como funcionam colunas para montagem dos componentes eletrônicos.

{\bf Materiais utilizados:} Multímetro, protoboard, bateria, LEDs e resistores.


\textbf{Atividade a ser desenvolvida: }

\Large
\textbf{Teoria: }

\normalsize
\begin{itemize}
    
    \item Fazer uma breve introdução sobre a importância da protoboard nos projetos eletrônicos. Com ela é possível montar, desenvolver e testar circuitos eletrônicos sem a necessidade de soldar os componentes.
    
    \item  Mostrar o funcionamento da protoboard, explicando as áreas que compõe os barramentos de alimentação, montagem de circuitos integrados e montagem de componentes.
    \item  Explicar como são montados os circuitos utilizando o multímetro para identificar como são interligados os barramentos.
    \item Mostrar os modelos de protoboards existentes no mercado.
    
\end{itemize}


\Large
\textbf{Prática: }

\normalsize
\begin{itemize}
    \item Identificar como os barramentos estão interligados utilizando o multímetro.
    \item Medir tensão e corrente elétrica com uma bateria conectada na protoboard.
    \item Medir a resistência elétrica de resistores conectados na protoboard.

\end{itemize}
    
{\Large
\textbf{Atividades práticas:}
}

	
Distribua as protoboards e os multímetros para os alunos e peça para que coloquem a chave seletora na posição continuidade e, com as pontas de prova testem dois a dois quais “furos” estão interligados (Figura \ref{fig:bagattoli_6}).


\begin{figure}[h!]
    \centering
    \includegraphics[width=0.65\textwidth]{capitulos/bagatolli_girardi/figuras/6.jpg}
    \caption{Identificando como os barramentos estão interligados.}
    \label{fig:bagattoli_6}
\end{figure}


Peça para os alunos conectarem alguns resistores na protoboard e com o multímetro, determine suas resistências (Figura \ref{fig:bagattoli_7}). Não oriente como devem colocar os resistores e verifique se compreenderam o funcionamento da matriz de contato da protoboard.


\begin{figure}[h!]
    \centering
    \includegraphics[width=0.65\textwidth]{capitulos/bagatolli_girardi/figuras/7.png}
    \caption{Medindo resistência elétrica.}
    \label{fig:bagattoli_7}
\end{figure}

\subsubsection{
Montagem de circuitos simples
}

{\bf Tempo estimado:} 120 minutos divididos entre 60 minutos de teoria e 60 minutos de prática.


{\bf Objetivo:} Compreender o funcionamento de circuitos simples e como podem ser conectados em uma protoboard. Explicar o funcionamento de circuitos utilizando uma bateria e um resistor, circuitos com amperímetro e voltímetro e circuitos em série e paralelo.


{\bf Materiais utilizados:} Multímetro, protoboard, bateria, LEDs e resistores.


{\bf Atividade a ser desenvolvida: }

{\Large
Teoria:
}

\begin{itemize}
    
    
    \item Explicar as características dos circuitos simples e como conecta-los utilizando a protoboard.
    \item Explicar as características das associações de resistores em série e em paralelo em um circuito.
    \item Explicar como é conectado o voltímetro (em paralelo) e o amperímetro (em série a um circuito).
    \item Explicar que o LED é polarizado e que só deverá acender se o terminal positivo de LED estiver conectado com o positivo da bateria e o terminal negativo com o negativo da bateria.
    \item Explicar como calcular o valor do resistor para ser associado ao LED no circuito para liga-lo sem queimar.
\end{itemize}
    
{\Large
Prática:
}

\begin{itemize}
    
    \item Montar um circuito simples utilizando uma bateria, um LED e um resistor na protoboard. Esse resistor deve ser determinado a partir das especificações do LED escolhido.
    \item Montar um circuito em série com dois resistores na protoboard e, com o auxílio do multímetro, demonstrar que a resistência equivalente na associação em série é igual a soma das resistências de cada resistor.
    \item Montar um circuito em paralelo com dois resistores na protoboard e, com o auxílio do multímetro, demonstrar que a resistência equivalente é igual à soma dos inversos da resistência de cada resistor associado.
    \item Montar um circuito utilizando um voltímetro e um amperímetro associado, medindo a tensão e a corrente do circuito simples.
\end{itemize}



{\bf Atividades práticas:}


Distribua para cada aluno um conjunto com uma protoboard, LEDs de cores variadas, uma bateria, um conjunto de resistores com resistências variadas e um multímetro. Peça que eles montem um circuito simples, composto de uma bateria (ou utilizar a saída 5V do Arduino), um resistor e um LED (Figura \ref{fig:bagattoli_8}). Eles deverão calcular o valor da resistência do resistor que será associado baseado nas características da fonte e do LED. Se necessário, peça aos alunos que façam um esboço do circuito, para depois montá-lo definitivamente na protoboard.

\begin{figure}[h!]
    \centering
    \includegraphics[width=0.65\textwidth]{capitulos/bagatolli_girardi/figuras/8.png}
    \caption{Circuito simples com um LED, um resistor e uma bateria.}
    \label{fig:bagattoli_8}
\end{figure}

Peça para que os alunos invertam a ordem da montagem (Figura \ref{fig:bagattoli_9}), verifiquem o que acontece e tentem explicar o fenômeno.


\begin{figure}[h!]
    \centering
    \includegraphics[width=0.65\textwidth]{capitulos/bagatolli_girardi/figuras/9.png}
    \caption{Circuito simples com um LED, um resistor e uma bateria invertidos.}
    \label{fig:bagattoli_9}
\end{figure}

Após terminar a montagem do circuito simples, prossiga com a montagem das associações em série e paralelo (Figura \ref{fig:bagattoli_10}), fazendo o esboço (se necessário), calculando a resistência equivalente, medindo e comparando os resultados. Para que os alunos não precisem calcular a resistência equivalente, pode ser utilizado o ElectroDroid (disponível em \url{https://play.google.com/store/apps/details?id=it.android.demi.elettronica&hl=pt_BR}) (Figura \ref{fig:bagattoli_11}).


\begin{figure}[h!]
    \centering
    \includegraphics[width=0.65\textwidth]{capitulos/bagatolli_girardi/figuras/10.png}
    \caption{ Exemplos de associação em série (a), paralelo (b) e mista (c).}
    \label{fig:bagattoli_10}
\end{figure}

\begin{figure}[h!]
    \centering
    \includegraphics[width=0.65\textwidth]{capitulos/bagatolli_girardi/figuras/11.png}
    \caption{ Cálculo da resistência equivalente através do código de cores\\
Fonte:\url{https://play.google.com/store/apps/details?id=it.android.demi.elettronica&hl=pt_BR}.}
    \label{fig:bagattoli_11}
\end{figure}


Após medir a resistência equivalente em cada associação, os alunos deverão montar um circuito com associações em série, paralelo ou mista (Figura 15) com o multímetro associado em paralelo, para medir tensão, e em série, para medir a corrente elétrica.

\begin{figure}[h!]
    \centering
    \includegraphics[width=0.65\textwidth]{capitulos/bagatolli_girardi/figuras/12.jpg}
    \caption{ Multímetro associado em paralelo e em série no circuito, respectivamente.}
    \label{fig:bagattoli_12}
\end{figure}


\subsection{Segunda etapa: Introdução ao Arduino e à linguagem de programação}

\subsubsection{
O que é Arduino?
}

{\bf Tempo estimado:} 30 minutos de teoria.
\\
{\bf Objetivo:} Conhecer um pouco sobre a origem do Arduino e compreender suas funcionalidades. 
\\
{\bf Materiais utilizados:} Placa Arduino.


{\bf Atividade a ser desenvolvida:}

{\Large
Teoria:
}

\begin{itemize}
    \item Fazer uma breve introdução sobre o que é um microcontrolador embarcado e sua utilização em projetos de automação.
    \item Explicar para que servem os conectores de alimentação, dando ênfase aos que serão utilizados inicialmente no projeto (3,3V, 5V, GND).
    \item Explicar para que servem as portas de entrada e saída, digital e analógico, destacando a diferença entre uma porta digital e uma porta analógica.
\end{itemize}

\subsection{
Conectando o Arduino no computador
}

{\bf Tempo estimado:} 30 minutos divididos entre 15 minutos de teoria e 15 minutos de prática.
\\
{\bf Objetivo:} Fazer a conexão da placa Arduino com o computador e realizar as configurações iniciais.
\\
{\bf Materiais utilizados:} Arduino, cabo USB.
\\
{\bf Atividade a ser desenvolvida:}
\\
{\Large
Teoria:
}

\begin{itemize}
    \item Explicar como é feita a conexão e o envio de dados do computador para o Arduino, através do cabo USB.
\end{itemize}

{\Large
Prática:
}

\begin{itemize}
    \item    Conectar a placa Arduino via USB no computador e fazer as configurações iniciais (selecionar a placa e a porta de comunicação na IDE do Arduino).
\end{itemize}

\subsubsection{Introdução à linguagem de programação}


{ Tempo estimado: } 60 minutos de atividades práticas.



{\bf Objetivo:} Compreender conceitos básicos de linguagem de programação.
\\

{\bf Materiais utilizados:} Arduino, computador.
\\

{\bf Atividade a ser desenvolvida: }
\\
{\Large
Prática:
}

\begin{itemize}
    \item     Montar um circuito utilizando um LED conectado em uma porta digital para explicar a utilização dos comandos “void setup” e “void loop”. Mostrar que o LED conectado em um pino digital só poderá assumir dois valores lógicos, HIGH (5V) ou LOW (0V). Exemplo de programação:
\end{itemize}

\begin{lstlisting}[language=C++]
void setup() {
  pinMode (13,OUTPUT);
}
 
void loop() {
  digitalWrite (13,HIGH);
  delay (2000);
  digitalWrite (13,LOW);
  delay (2000);
}
\end{lstlisting}

\begin{itemize}
    
    \item {\bf void setup( )} é uma função e é executada uma vez assim que o arduino é ligado. 
    \item {\bf void loop( )} também é uma função que é executada, como diz o próprio nome, em “loop”, enquanto o Arduino estiver ligado.
    \item O comando {\bf pinMode (13, OUTPUT)} define o pino digital 13 do Arduino como um pino de saída (output). É o pino em que o LED está ligado.
    \item O comando {\bf digitalWrite(13,HIGH)} liga o LED.
    \item O comando {\bf digitalWrite(13,LOW)} desliga o LED.
    \item O comando {\bf delay(2000)} faz o Arduino esperar dois segundos antes de executar o próximo comando.
\end{itemize}

{\bf Atividades práticas:}

Monte um circuito com um resistor e um LED (Figura \ref{fig:bagattoli_13}), utilizando a porta digital 11. Importante observar que o LED conectado em uma porta digital só poderá assumir um de dois valores lógicos possíveis, ligado (5V) ou desligado (0V).


\begin{figure}[h!]
    \centering
    \includegraphics[width=0.65\textwidth]{capitulos/bagatolli_girardi/figuras/13.png}
    \caption{ Circuito com um resistor ($220\Omega$) e um LED.}
    \label{fig:bagattoli_13}
\end{figure}


Abrindo a IDE do Arduino, copie e cole o exemplo de programação:
\begin{lstlisting}[language=C++]
    void setup() {
        pinMode (11,OUTPUT);
        }
        
        void loop() {
            digitalWrite (11,HIGH);
            delay (2000);
            digitalWrite (11,LOW);
            delay (2000);
            }
        \end{lstlisting}
        
Para carregar o código para a placa, clique na seta (Figura \ref{fig:bagattoli_14}).
        
        \begin{figure}[h!]
            \centering
            \includegraphics[width=0.65\textwidth]{capitulos/bagatolli_girardi/figuras/14.png}
            \caption{ Carregando o código para a placa Arduino.}
            \label{fig:bagattoli_14}
        \end{figure}

\begin{itemize}
    \item {\bf void setup( )} é um método e é executado uma vez assim que o arduino é ligado. 
    \item {\bf void loop( )} também é um método que é executado, como diz o próprio nome, em loop enquanto o Arduino estiver ligado.O comando {\bf pinMode (11, OUTPUT)} define o pino digital 11 do Arduino como um pino de saída. É o pino em que o LED está ligado.
    \item O comando {\bf digitalWrite(11,HIGH)} liga o LED.
    \item O comando {\bf digitalWrite(11,LOW)} desliga o LED.
\end{itemize}

O comando {\bf delay(2000)} faz o Arduino esperar dois segundos antes de executar o próximo comando. Peça para os alunos alterar os valores do delay (3000, 100, 50, ...), mudar o LED de porta, acrescentar LEDs utilizando o mesmo resistor e façam alterações na programação.

\subsubsection{Utilizando o Ardublock para programar}


{\bf Tempo estimado:} 120 minutos de atividades práticas.


{\bf Objetivo:} Compreender conceitos básicos de linguagem de programação, utilizando o Ardublock.


{\bf Materiais utilizados:} Arduino, computador.



{\bf Atividade a ser desenvolvida: }

{\Large
Prática:
}
\begin{itemize}
    \item     Montar um circuito utilizando um LED conectado em uma porta digital para explicar a utilização dos comandos (blocos) no Ardublock (Figura \ref{fig:bagattoli_15}).
\end{itemize}
    
\begin{figure}[h!]
            \centering
            \includegraphics[width=0.65\textwidth]{capitulos/bagatolli_girardi/figuras/15.png}
            \caption{ Estrutura de programação utilizando o Ardublock.}
            \label{fig:bagattoli_15}
        \end{figure}

        \begin{itemize}
            \item Sempre faça: é um método que é executado em “loop” enquanto o Arduino estiver ligado.
            \item O bloco seta pino digital (D13,HIGH) liga o LED.
            \item O bloco delay MILLIS milissegundos (2000) faz o Arduino esperar dois mil milissegundos (ou 2 segundos) antes de executar o próximo comando.
            \item O bloco seta pino digital (D13,LOW) desliga o LED.
        \end{itemize}

{\bf Atividades práticas:}

Monte um circuito com um resistor e um LED (Figura \ref{fig:bagattoli_15}), utilizando a porta digital 11. Para acender o LED, utilizaremos uma estrutura básica de programação no Ardublock, seguindo a mesma lógica de programação já utilizada para este exemplo. Para facilitar o uso, os blocos possuem as mesmas cores do menu. Por exemplo, o bloco “sempre faça” tem a mesma cor da opção “controle” (podem ocorrer algumas variações dependendo da versão utilizada). Para fazer o upload do programa para a placa, clique em “enviar para o Arduino”. É importante observar que ao carregar a programação feita no Ardublock para o Arduino, as linhas de comando do respectivo código aparecem na IDE e serve como objeto de estudo e pode inclusive ser alterado.


Peça para os alunos alterar os valores do delay (3000, 100, 50, ...), mudar o LED de porta, acrescentar LEDs utilizando o mesmo resistor e façam alterações na programação.

% TERCEIRA ETAPA
\subsection{Terceira etapa: Compreendendo e coletando os dados do arduino}

\subsubsection{
Compreendendo o sistema divisor de tensão e fazendo o tratamento dos dados obtidos com o Arduino
}

{\bf Tempo estimado:} 240 minutos divididos entre 60 minutos de teoria e 180 minutos de prática.


{\bf Objetivo:} Compreender o funcionamento do divisor de tensão e utilizá-lo para fazer a leitura dos resistores utilizados na oficina, fazendo o tratamento dos dados obtidos.


{\bf Materiais utilizados:} Arduino, computador, LDR, LED, protoboard, resistores e termistores PTC/NTC.


{\bf Atividade a ser desenvolvida: }

{\Large
Teoria:
}
\begin{itemize}
    \item  Explicar como funciona um sistema divisor de tensão através da associação de dois resistores em série, com o objetivo de se obter uma determinada fração da tensão. Mostrar que a tensão nos terminais de cada resistor é proporcional a sua resistência elétrica e que pode ser calculado utilizando a seguinte relação..
\end{itemize}
\begin{equation}
    U_{\text{saída}} = \frac{R2}{R1+R2} U_{\text{entrada}}
    \nonumber
\end{equation}



{\Large
Prática:}

\begin{itemize}
    
    \item Montar um sistema divisor de tensão utilizando um LDR e um resistor de $10k\Omega$.
    \item Montar um sistema divisor de tensão utilizando um termistor e um resistor de $10k\Omega$, utilizando o exemplo acima.
    \item Para fazer a leitura dos dados captados pelo LDR/Termistor (Figura \ref{fig:bagattoli_16}), utilize o “monitor serial”.
    
    \begin{figure}[h!]
        \centering
        \includegraphics[width=0.65\textwidth]{capitulos/bagatolli_girardi/figuras/16.png}
        \caption{ Estrutura de programação para fazer a leitura dos dados pelo monitor serial.}
        \label{fig:bagattoli_16}
    \end{figure}
    
    \item Calibrar os sensores (Lux, °C), encontrando uma equação de transformação dos dados obtidos nos sensores para as unidades 
\end{itemize}
    

{\bf Atividades práticas:}


Monte um sistema divisor de tensão utilizando um LDR e um resistor de $10k\Omega$ (Figura \ref{fig:bagattoli_17}). Com o multímetro, mostre aos alunos que o valor total da tensão (5V) é igual a soma da tensão de cada resistor. Monte um sistema divisor de tensão utilizando também um termistor.

    \begin{figure}[h!]
        \centering
        \includegraphics[width=0.65\textwidth]{capitulos/bagatolli_girardi/figuras/17.png}
        \caption{ Sistema divisor de tensão com um LDR e um resistor de $10k\Omega$.}
        \label{fig:bagattoli_17}
    \end{figure}
    
Para fazer a leitura dos dados obtidos no LDR, utilizaremos o monitor serial do Arduino, que se encontra no canto superior direito da IDE (Figura \ref{fig:bagattoli_18}).

    \begin{figure}[h!]
        \centering
        \includegraphics[width=0.65\textwidth]{capitulos/bagatolli_girardi/figuras/18.png}
        \caption{ Localização do monitor serial do Arduino.}
        \label{fig:bagattoli_18}
    \end{figure}
    

No monitor serial aparecerão dados obtidos pelos sensores em uma escala de 0 a 1023 sendo 0 (0V) e 1023 (5V) (Figura \ref{fig:bagattoli_19}). Qualquer valor entre 0 e 1023 terá um valor proporcional entre 0V e 5V. Para fazer o tratamento desses dados será necessário calibrar os sensores e encontrar equações de conversão para as grandezas físicas em questão (temperatura, luminosidade, etc.).
 \begin{figure}[h!]
        \centering
        \includegraphics[width=0.65\textwidth]{capitulos/bagatolli_girardi/figuras/19.png}
        \caption{ Dados obtidos no monitor serial.}
        \label{fig:bagattoli_19}
    \end{figure}
    
    
Faça a programação no Ardublock, utilizando a função “imprime” (Figura 24) e junto com o pino analógico A0 (de acordo com o exemplo de montagem).

\begin{figure}[h!]
           \centering
           \includegraphics[width=0.65\textwidth]{capitulos/bagatolli_girardi/figuras/20.png}
           \caption{ Exemplo de programação utilizando Ardublock.}
           \label{fig:bagattoli_20}
       \end{figure}


\subsection{Quarta etapa: Construindo uma estação meteorológica com os resistores utilizados na oficina}

\subsubsection{Construindo uma estação meteorológica com os resistores utilizados na oficina}

{\bf Tempo estimado:} 240 minutos de atividades práticas.


{\bf Objetivo:} Montar uma estação meteorológica em equipe, aplicando as habilidades desenvolvidas durante a oficina.


{\bf Materiais utilizados:} Arduino, computador, LDR, LED, multímetro, motor DC – 5V, protoboard, resistores e termistores PTC/NTC.

{\bf
Atividade a ser desenvolvida: 
}

Os alunos deverão construir uma estação meteorológica com os sensores utilizados e calibrados na oficina. Durante as etapas, os grupos já montaram individualmente um sensor de luminosidade e um sensor de temperatura, encontrando suas equações características. Podem utilizar o motor 5V DC com um circuito divisor de tensão para construir um anemômetro. O desafio será montar e programar todos os sensores em conjunto (Figura \ref{fig:bagattoli_21}), de moto que os dados possam ser observados pelo monitor serial e, se disponível, em um LCD. Para organizar os circuitos, pode-se utilizar uma caixa para disjuntores, que possui uma abertura para fixação do LCD (Figura \ref{fig:bagattoli_22}).

\begin{figure}[h!]
           \centering
           \includegraphics[width=0.65\textwidth]{capitulos/bagatolli_girardi/figuras/21.jpg}
           \caption{ Sensor de luz, temperatura e velocidade.}
           \label{fig:bagattoli_21}
       \end{figure}
\begin{figure}[h!]
           \centering
           \includegraphics[width=0.65\textwidth]{capitulos/bagatolli_girardi/figuras/22.jpg}
           \caption{ Caixa para disjuntor.}
           \label{fig:bagattoli_22}
       \end{figure}


Nesse momento, espera-se que os alunos percebam a necessidade de se implementar outros sensores, como por exemplo: sensor de nível d’água, sensor de umidade, display LCD entre outros.

